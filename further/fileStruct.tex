\indent\indent The main information on each correlation process is stored in CSV format. For each analysis, the following files will be created:
\begin{itemize}
  \item \textit{filenamelist.csv} : This file contains names of your images files
  \item \textit{gridx.csv} - \textit{gridy.csv} : These files store respectively the x and y coordinates of each marker in your grids along with their instance number.
  \item \textit{validx.csv} - \textit{validy.csv} : These files store respectively the x and y coordinates of each marker of your analysis. Each line corresponds to a marker, each column corresponds to an image.
  \item \textit{dispx.csv} - \textit{dispy.csv} : These files store respectively the x and y displacements in pixel of each marker of your analysis. Each line corresponds to a marker, each column corresponds to an image.
  \item \textit{stdx.csv} - \textit{stdy.csv} : These files store respectively the x and y standard deviation in pixel of each marker of your analysis. Each line corresponds to a marker, each column corresponds to an image.
  \item \textit{corrcoef.csv} : This file stores the correlation coefficient or each marker of your analysis. Each line correspond to a marker, each column corresponds to an image.
  \item \textit{neighbors.csv} : This file stores the closest neighbors of each marker. Each line correspond to a marker. This file is updated when neighbors are re-calculated.
  \item \textit{infoMarkers.csv} and \textit{infoAnalysis.csv} : These files store informations on your analysis such as correlation errors and parameters used.
  \item \textit{strainx.csv} and \textit{strainy.csv} : These files store the global strain values of each grid along images. Each line corresponds to the image, each column correspond to a grid. These two files are automatically updated when coordinates are re-calculated.
  \item \textit{coordinates.csv} (After first start) : This file contains calculated coordinates for the 2D mapped plots. They're stored to save time during start-up and may not be up-to-date if the user decided not to re-calculate them after the cleaning process. An option is available in the \textit{More} menu to re-calculate these coordinates at any time. 
  \item \textit{filter.csv} (Optional) : This file stores information relative to filters applied.
  \item \textit{largeDisp.csv} (Optional) : This file stores the calculated large displacement on your analysis in case the shift correction have been used.
\end{itemize}

\paragraph{Masks\\\newline}
\label{par:Masks}
\newline
\indent Masks (or versions) are saved in the \textit{log} folder of each analysis. This folder is automatically created when the first mask is applied. A mask is then saved whenever markers or images are hide on the analysis. You can always load an old version of your analysis by importing one of these masks.
